% -*- coding: utf-8; -*-

\chapter{Revisão Bibliográfica}

This is the second chapter...

In this chapter, let's have a nice table:

%% -*- coding: utf-8; -*-

\begin{table} [!h]
  \caption{Principais minerais de ferro e suas classes.\cite{29}}\label{tab:2-4}
  ~\\[-1mm]
   \begin{tabularx}
     {\textwidth}
     { p{2.0cm}
       p{2.5cm}
       p{3.3cm}
       p{1.3cm}
       p{2.7cm}}

     \textbf{Classes}
     & \textbf{Minerais}
     & \textbf{\mrcel {Fórmula}{Química}}
     & \textbf{\mrcel{Teor}{de Fe}}
     & \textbf{\mrcel{~~Designação}{~~Comum}}
     \\\toprule

     ~ \\[-6mm]
     \multirow{5}{*}{Óxidos}& Magnetita
     & $Fe_{3}O_{4}$
     & ~72,4
     & \mrcel{~~Óxido ferroso}{~~férrico}
     \\%\midrule

     & Hematita
     & $Fe_{2}O_{3}$
     & ~69,9
     & ~~Óxido férrico \\[2mm]

     & Goethita
     & $FeO(OH)$
     & ~62,8
     & \multirow{2}{*}{\mrcel{Óxido-hidróxido}{de ferro}} \\[2mm]


     & Lepidocrocita
     & $FeO(OH)$
     & ~62,8 &
     \\\midrule

     Carbonato
     & Siderita
     & $FeCO_{3}$
     & ~48,2
     & \mrcel{~~~~Carbonato}{~~~~de Ferro}
     \\\midrule

     \multirow{2}{*}{Sulfetos}
     & Pirita
     & $FeS_{2}$
     & ~46,5
     & \multirow{2}{*}{~} \\[2mm]


     & Pirrotita
     & $FeS$
     & ~63,6
     & ~
     \\\midrule

     \multirow{10}{*}{Silicatos}
     & Fayalita
     & $Fe^{2+}_{2}(SiO_{4})$
     & ~54,8
     & \mrcel{~~~~Grupo da}{~~~~Olivina} \\[4mm]

     & Laihunite
     & $Fe^{2+}Fe^{3+}_{2}(SiO_{4})_{2}$
     & ~47,6
     & \mrcel{~~~~Grupo da}{~~~~Olivina} \\[4mm]

     & Greenalita
     & \mrcell{$2Fe^{2+}_{2}6Fe^{3+}Si_{2}$}{$4O_{5}(OH)_{3,3}$}
     & ~44,1
     & \mrcel{~~~~Grupo da}{~~~~Serpentina} \\[4mm]

     & Grunerita
     & \mrcell{$Fe^{2+}_{7}(Si_{8}O_{22})$}{$(OH)_{2}$}
     & ~39,0
     & \mrcel{~~~~Grupo dos}{~~~~Anfibólios} \\[4mm]

     & Fé-antofilita
     & \mrcell{$Fe^{2+}_{7}(Si_{8}O_{22})$}{$(OH)_{2}$}
     & ~39,0
     & \mrcel{~~~~Grupo dos}{~~~~Anfibólios}
     \\\midrule
   \end{tabularx}
\end{table}

% -*- coding: utf-8; -*-

\newcommand{\coltworowone}{%
\begin{tabular}{ l @{\extracolsep{2mm}}X }
  \mybulletOB
    & Cristais ~~ muito	\\[-1.2mm]
  ~ & pequenos, 	\\[-.4mm]
  ~ & $<0.01$ mm. \\
  \mybulletOB & Textura porosa. \\
  \mybulletOB
    & Contatos pouco	\\[-1.2mm]
  ~ & desenvolvidos.
\end{tabular}}

\newcommand{\coltworowtwo}{%
\begin{tabular}{ l @{\extracolsep{2mm}}X }
  \mybulletOB
    & Cristais euédricos \\[-1.2mm]
  ~ & isolados ~ ou ~ em \\[-1.2mm]
  ~ & agregados. \\
  \mybulletOB
    & Cristais compac- \\[-1.2mm]
  ~ & tos.
\end{tabular}}

\newcommand{\coltworowthree}{%
\begin{tabular}{ l @{\extracolsep{2mm}}X }
  \mybulletOB
    & Hematita ~~ com \\[-1.2mm]
  ~ & hábito de magne- \\[-1.2mm]
  ~ & tita. \\
  \mybulletOB
    & Oxidação segundo \\[-1.2mm]
  ~ & os planos ~ crista- \\[-1.2mm]
  ~ & lográficos da mag- \\[-1.2mm]
  ~ & netita. \\
  \mybulletOB
    & Geralmente ~ po- \\[-1.6mm]
  ~ & rosa.
\end{tabular}}

\newcommand{\coltworowfour}{%
\begin{tabular}{ l @{\extracolsep{2mm}}X }
  \mybulletOB
    & Formatos irregu- \\[-1.2mm]
  ~ & lares ~~ inequidi- \\[-1.2mm]
  ~ & mensionais. \\
  \mybulletOB
    & Contatos irregula-\\[-1.2mm]
  ~ & res, ~ geralmente \\[-1.2mm]
  ~ & imbricados.
 \end{tabular}}

\newcommand{\coltworowfive}{%
\begin{tabular}{ l @{\extracolsep{2mm}}X }
  \mybulletOB
    & Formatos regula- \\[-1.2mm]
  ~ & res ~~ equidimen- \\[-1.2mm]
  ~ & sionais. \\
  \mybulletOB
    & Contatos ~~ retilí- \\[-1.2mm]
  ~ & neos ~ e ~ junções \\[-1.2mm]
  ~ & tríplices. \\
  \mybulletOB
    & Cristais compac-\\[-1.6mm]
  ~ & tos.
 \end{tabular}}

\newcommand{\coltworowsix}{%
\begin{tabular}{ l @{\extracolsep{2mm}}X }
  \mybulletOB
    & Cristais inequidi- \\[-1.2mm]
  ~ & mensionais, hábi- \\[-1.2mm]
  ~ & to tabular. \\
  \mybulletOB
    & Contato retilíneo. \\
  \mybulletOB
    & Cristais compac- \\[-1.6mm]
  ~ & tos.
 \end{tabular}}

\newcommand{\coltworowseven}{%
\begin{tabular}{ l @{\extracolsep{2mm}}X }
  \mybulletOB
    & Material cripto- \\[-1.2mm]
  ~ & cristalino.  \\
  \mybulletOB
    & Estrutura colofor- \\[-1.2mm]
  ~ & me, hábito botri- \\[-1.2mm]
  ~ & oidal.  \\
  \mybulletOB
    & Textura porosa.
 \end{tabular}}

\begin{table} [!p]
    \caption{Quadro ilustrativo com as principais morfologias de cristais de óxidos/hidróxidos de ferro.\cite{14}}\label{tab:2-5}
    ~\\[-2mm]
  \begin{tabularx}{\textwidth}{@{\extracolsep{0pt}}C @{\extracolsep{0pt}}C C C}

    \textbf{Tipo}
    & \textbf{Características}
    & \textbf{\mrcel{Forma}{Textura}}
    & \textbf{\mrcel{Ilustração}{Esquemática}}
    \\\toprule

    ~ \\[-6mm]
    \mrcel{Hematita}{Microcristalina}
    & \coltworowone
    & \mytbcimg{2.3cm}{2.9cm}{images/Microcristalina}
    & \mytbcimg{2.6cm}{2.5cm}{images/MicrocristalinaEsq}
    \\\midrule

    ~\\[-6mm]
    Magnetita
    & \coltworowtwo
    & \mytbcimg{2.3cm}{2.9cm}{images/Magnetita}
    & \mytbcimg{2.6cm}{2.9cm}{images/MagnetitaEsq}
    \\\midrule

    ~\\[-5mm]
    Martita
    & \coltworowthree
    & \mytbcimg{2.3cm}{2.9cm}{images/Martita}
    & \mytbcimg{2.6cm}{2.9cm}{images/MartitaEsq}
    \\\midrule

    ~\\[-5mm]
    \mrcel{Hematita}{Lobular}
    & \coltworowfour
    & \mytbcimg{2.3cm}{2.9cm}{images/Lobular}
    & \mytbcimg{2.6cm}{2.9cm}{images/LobularEsq}
    \\\midrule

    ~\\[-5mm]
    \mrcel{Hematita}{Granular}
    & \coltworowfive
    & \mytbcimg{2.3cm}{2.9cm}{images/Granular}
    & \mytbcimg{2.6cm}{2.9cm}{images/GranularEsq}
    \\\midrule

    ~\\[-5mm]
    \mrcel{Hematita}{Lamelar}
    & \coltworowsix
    & \mytbcimg{2.3cm}{2.9cm}{images/Lamelar}
    & \mytbcimg{2.6cm}{2.9cm}{images/LamelarEsq}
    \\\midrule

    ~\\[-5mm]
    \mrcelthree{Hidróxidos de}{ Fe (Goethita-}{Limonita)}
    & \coltworowseven
    & \mytbcimg{2.3cm}{2.9cm}{images/Goethita}
    & \mytbcimg{2.6cm}{2.9cm}{images/GoethitaEsq}
    \\\midrule

  \end{tabularx}
\end{table}


\section{Hematita}

A hematita é o mineral de ferro mais importante devido a sua alta
ocorrência em vários tipos de rochas e suas origens diversas.\cite{30}
A composição química deste mineral é Fe$_{2}$O$_{3}$, com uma fração
mássica em ferro de 69,9\% e uma fração mássica em oxigênio de
30,1\%.\cite{31}

...


\subsection{Martita}

A hematita é o mineral de ferro mais importante devido a sua alta
ocorrência em vários tipos de rochas e suas origens diversas.\cite{30}
A composição química deste mineral é Fe$_{2}$O$_{3}$, com uma fração
mássica em ferro de 69,9\% e uma fração mássica em oxigênio de
30,1\%.\cite{31}

...


\subsubsection{Globular}

A hematita é o mineral de ferro mais importante devido a sua alta
ocorrência em vários tipos de rochas e suas origens diversas.\cite{30}
A composição química deste mineral é Fe$_{2}$O$_{3}$, com uma fração
mássica em ferro de 69,9\% e uma fração mássica em oxigênio de
30,1\%.\cite{31}

...
